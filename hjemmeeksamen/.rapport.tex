\documentclass[11pt,a4paper,norsk]{article}
\usepackage[norsk]{babel}
\usepackage[utf8]{inputenc}
\title{\textbf{Hjemmeeksamen rapport}}
\author{INF1060 H15}
\begin{document}
\maketitle

\section*{Oppgaver}
\subsection*{1)}
TCP protokollen gir en pålitelig overføring av data. TCP passer på at ingen pakker forsvinner, og 
at pakkene blir overført i riktig rekkefølge. Hvis en pakke ikke er mottatt på en bestemt tid, vil 
et tidsavbrudd oppstå. Sender vil da få beskjed om å sende pakken på nytt.

\subsection*{2)}
Jeg har en enkel tekst protokoll som bare overfører data umodifisert. I en ideell protokoll, er 
første byte antall byte hele pakken består av, n bytes etter den første er for data som skal 
overføres. Med denne protokollen vet mottaker hvor mange bytes pakken består av.

\subsection*{3)}
Det kan gå galt hvis man ikke bruker hton() og ntoh() for å få riktig byte rekkefølge.

%\subsection*{4)}


\subsection*{5)}
Det finnes fork, epoll, select. Jeg har brukt select, fordi det er den jeg har lært å bruke.

\end{document}